\documentclass{report}

\usepackage[left=3cm,right=3.5cm,top=2cm,bottom=2cm]{geometry}
\usepackage{graphicx}
\usepackage{caption}
\usepackage{subcaption}
\usepackage{tabularx}
\usepackage{booktabs}
\usepackage{multirow}
\usepackage{tablefootnote}
\usepackage{float}

\graphicspath{{../Plots/}}

\begin{document}
\renewcommand{\arraystretch}{1.5}

\pagenumbering{arabic}
\begin{flushright}
  \today
\end{flushright}

\begin{center}
  \huge Advanced Atomic and Molecular Physics\\ 
  \Large Lab course "Femtosecond Spectroscopy"\\
\end{center}

\begin{flushleft} 
  \large Erstellt von: Lukas Elflein, Fabian Thielemann, Lukas G\"otz, Chiara
  Lindner\\
\end{flushleft}

The solutions to the analytical tasks are handed in in handwritten form.

\subsection*{Answers to questions}
  \begin{itemize}
    \item Measuring the fluorescence light perpendicular to the laser beam
      reduces direct light, which results in a
      better signal to noise ratio.

    \item The FWHM of the two-pulse cross-correlation 
      becomes (see calculations)
      $$ \Delta t_{AC} = \Delta t$$

    \item For constant inter pulse delay $\tau$ the two-pulse
      cross-correlation becomes
      $$ S^{pm}_{PD} \propto \underbrace{e^{-4\ln2\left(\frac{\tau}{\Delta
        t_{AC}}\right)}}_{cst.} \cos(\underbrace{\omega_L
      \tau}_{cst.}  - \Omega_{21}T_m )$$
      so for the given frequencies the cosine with angular frequency 5\,kHz
      will be screened with a frequency of 80\,MHz. Changing $\tau$ will
      change the amplitude as well as the phase of the cosine.

    \item Higher order perturbation calculation describe higher numbers of
      photons involved in the transition. $S^{(2)}_F$ describes single photon

      transitions, $S^{(4)}_F$ describes two photon transitions.
    \item The fluorescence signals of $S^{(2)}_F$ and $S^{(4)_F}$ can be
      distinguished by their modulation frequencies which are $\Omega_{21}$
      and $2\Omega_{21}$ respectively

    \item The demodulated signal will be a sinoidal oscillation with angular
      frequency $\omega_{eg} - \omega_M$ in the case of single photon
      transitions
      
    \item The reference signal for the 4th order signal would need to be
      modulated with the frequency $2 \Omega_{21}$. This could be obtained by
      frequency doubling of the reference signal of the 2nd order process.

    \item Using the AOMs combined with the lock-in amplifier one can select
      the order of the monitored interaction. In this way the measured signal
      consists of less superposed frequencies and can be analysed more easily.
  \end{itemize}
  
\subsection*{Results}
\subsubsection*{FWHM of pulses}
  To determine the FWHM of the two-pulse cross-correlation, first the delay
  stage was moved to the position of the maximum amplitude of $S^{pm}_{PD}$. Then
  the delay stage was move until the half of the maximum amplitude was reached.
  From this we obtained the FWHM as
  $$\Delta t_{FWHM} = 2 \frac{\Delta d}{c} = 226\,\textrm{fs},  $$
  where $c$ is the speed of light. Using the same method for the light pulses
  after the monochromator, we get a FWHM of
  $$ \Delta t_{FWHM}^{mono} = 41\,\textrm{ps}. $$
  This broadening is due to a narrowing in the frequency space by the
  monochromator.

  The spectral FWHM of the laser was measured directly with a spectrometer
  although the spectrum was not perfectly symmetric. From
  this we obtained a value of 
  $$ \Delta \nu = 1.78\,\textrm{THz} $$.
  From this measurements we can calculate the value of
  $$\Delta\nu\Delta\t = 0.405$$
  with is a good approximation to the ideal value of 0.441 for gaussian pulses.
\subsubsection*{Spectroscopy of Rb}
For all measured transitions, the full (complex) signal was reconstructed from the data
 by $Z = X + iY$. The obtained signal was then transformed to the frequency
 space using a fast fourier transform (FFT) implemented in \texttt{numpy}.  The
 monochromator wavelenth of 774.45\,nm was transformed to a frequency and added
 to the frequencies obtained from the FFT. The results for Rb are shown
 in Fig.~\ref{fig:Rb_fft}. For the first order processes we can see one clear
 peak, for the fourth order processes there are three. Note that the small peaks
 on the right are the suppressed sine-cosine-mirror images of the peaks on
 the left side.
 \begin{figure}[H]
   \centering
   \begin{subfigure}[b]{0.45\textwidth}
     \includegraphics[width=\textwidth]{fft_Rb_1gamma.pdf}
      \caption{Second order fluorescence signal}     
   \end{subfigure}
   \quad
   \begin{subfigure}[b]{0.45\textwidth}
    \includegraphics[width=\textwidth]{fft_Rb_2gamma.pdf}  
    \caption{Fourth order fluorescence signal}
   \end{subfigure}
   \caption{FFT of the signals obtained from the spectroscopy of Rb. The
   monochromator frequency has been added to the frequency axis.}
   \label{fig:Rb_fft}
 \end{figure}
 The results from the FFT analysis are shown in the table below. The transition
 $5S_{1/2} \rightarrow 5P_{3/2}$ was identified in the single photon transition
 signal. The obtained wavelength is in good accordance with the value from the
 NIST databank. In the fourth order signal, the middle peak could be identified with the 
$5S_{1/2} \rightarrow 5D_{3(5)/2}$ transition. The leftmost peak in the fourth order spectrum
has half the wavelength of the $5S_{1/2} \rightarrow 5P_{3/2}$ transition, it
is thus probably a second harmonic of it. The third peak could not be assigned
to a known transition in the spectrum of Rb. The differences in amplitude of the
peaks can be explained by the prefactors of the fluorescence signals $S_F$.
Different transition moments $\mu$ will result in different amplitudes of the
peaks in the fourier spectrum as they do not change under
transformation.
 \begin{table}[H]
   \centering
   \caption{Results obtained from the FFT analysis of the Rb signals}
   \label{tab:Rb_results}
   \begin{tabular}{ccccccc}
      \toprule
      & & $\nu$ / THz  & $\lambda$ / nm  & $\lambda_{Theo}$\tablefootnote{Values taken
    from NIST Databank \texttt{http://www.nist.gov/pml/data/asd.cfm}}  & Transition\\
      \midrule
      1$\gamma$ & Peak 1  & 384.24 & 780.21  & 780.24 &
      $5S_{1/2} \rightarrow 5P_{3/2} $\\
      \midrule
      \multirow{3}{*}{2$\gamma$} & Peak 1 & 768.45 & 390.12 & 390.12 & $5S_{1/2}
      \rightarrow 5P_{3/2}$ (SHG)\\
                                 & Peak 2 & 770.58  & 389.04 & 389.05 (389.09) &
      $5S_{1/2} \rightarrow 5D_{3(5)/2}$\\
                                 & Peak 3 & 771.34 & 388.66 & - & \\
      \bottomrule
  \end{tabular}
 \end{table}
\subsubsection*{Spectroscopy of Cs}
 For the analysis of the $6S_{1/2} \rightarrow 8S_{1/2}$ transition of Cs, the
 fourth order signal at a monochromator wavelength of 822.33\,nm was treated
 similarily to the analysis of the Rb spectrum. The FFT of the signal showed one
 clear peak at a frequency of $\nu_{Cs}=729.22$\,THz (see Fig.~\ref{fig:Cs_fft}). This corresponds to a
 wavelength of 
 $$\lambda_{Cs} =  411.10\,\textrm{nm},$$
 a value that is in good accordance with the value from the NIST databank of
 $\lambda_{theo}^{Cs} = 411.23\,\textrm{nm}$ for the $6S_{1/2} \rightarrow
 8S_{1/2}$ transition. 
 \begin{figure}[H] 
   \centering
     \includegraphics[width=0.9\textwidth]{fft_Cs_2gamma.pdf}
     \caption{FFT of the fourth order fluorescence signal of Cs. The
     monochromator frequency is added to the frequency axis.}
     \label{fig:Cs_fft}
 \end{figure}
\end{document}
