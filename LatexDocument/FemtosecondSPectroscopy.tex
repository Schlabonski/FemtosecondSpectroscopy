\documentclass{report}

\usepackage[left=2.5cm,right=2.5cm,top=2cm,bottom=2cm]{geometry}
\usepackage{graphicx}
\usepackage{caption}
\usepackage{subcaption}
\usepackage{tabularx}
\usepackage{booktabs}
\usepackage{multirow}

\graphicspath{{../Plots/}}

\begin{document}
\renewcommand{\arraystretch}{1.5}

\pagenumbering{arabic}
\begin{flushright}
  \today
\end{flushright}

\begin{center}
  \huge Advanced Atomic and Molecular Physics\\ 
  \Large Lab course "Femtosecond Spectroscopy"\\
\end{center}

\begin{flushleft} 
  \large Erstellt von: Lukas Elflein, Fabian Thielemann, Lukas G\"otz, Chiara
  Lindner\\
\end{flushleft}

The solutions to the analytical tasks are handed in in handwritten form.

\subsection*{Answers to questions}
  \begin{itemize}
    \item Measuring the fluorescence light perpendicular to the laser beam
      reduces direct light, which results in a
      better signal to noise ratio.

    \item The FWHM of the two-pulse cross-correlation 
      becomes (see calculations)
      $$ \Delta t_{AC} = \Delta t$$

    \item For constant inter pulse delay $\tau$ the two-pulse
      cross-correlation becomes
      $$ S^{pm}_{PD} \propto \underbrace{e^{-4\ln2\left(\frac{\tau}{\Delta
        t_{AC}}\right)}}_{cst.} \cos(\underbrace{\omega_L
      \tau}_{cst.}  - \Omega_{21}T_m )$$
      so for the given frequencies the cosine with angular frequency 5\,kHz
      will be screened with a frequency of 80\,MHz. Changing $\tau$ will
      change the amplitude as well as the phase of the cosine.

    \item Higher order perturbation calculation describe higher numbers of
      photons involved in the transition. $S^{(2)}_F$ describes single photon

      transitions, $S^{(4)}_F$ describes two photon transitions.
    \item The fluorescence signals of $S^{(2)}_F$ and $S^{(4)_F}$ can be
      distinguished by their modulation frequencies which are $\Omega_{21}$
      and $2\Omega_{21}$ respectively

    \item The demodulated signal will be a sinoidal oscillation with angular
      frequency $\omega_{eg} - \omega_M$ in the case of single photon
      transitions
      
    \item The reference signal for the 4th order signal would need to be
      modulated with the frequency $2 \Omega_{21}$. This could be obtained by
      frequency doubling of the reference signal of the 2nd order process.

    \item Using the AOMs combined with the lock-in amplifier one can select
      the order of the monitored interaction. In this way the measured signal
      consists of less superposed frequencies and can be analysed more easily.
  \end{itemize}
  
\subsection*{Results}
For all measured transitions, the full (complex) signal was reconstructed from the data
 by $Z = X + iY$. The obtained signal was then transformed to the frequency
 space using a fast fourier transform (FFT) implemented in \texttt{numpy}. The results for Rb are shown
 in \ref{fig:Rb_fft}. For the single photon transitions we can see one clear
 peak, for the two photon transitions there are three. Note that the small peaks
 on the right are the suppressed sine-cosine-mirror images of the peaks on
 the left side.
 \begin{figure}[h]
   \centering
   \begin{subfigure}[b]{0.45\textwidth}
     \includegraphics[width=\textwidth]{fft_Rb_1gamma.pdf}
      \caption{Single photon transitions}     
   \end{subfigure}
   \quad
   \begin{subfigure}[b]{0.45\textwidth}
    \includegraphics[width=\textwidth]{fft_Rb_2gamma.pdf}  
    \caption{Two photon transitions}
   \end{subfigure}
   \caption{FFT of the signals obtained from the spectroscopy of Rb. The
   monochromator frequency has already been added to the frequency axis.}
   \label{fig:Rb_fft}
 \end{figure}
 The results from the FFT analysis are shown in the table below.
 \begin{table}[h]
   \centering
   \caption{Results obtained from the FFT analysis of the Rb signals}
   \label{tab:Rb_results}
   \begin{tabular}{ccccccc}
      \toprule
      & & $\nu$ / THz  & $\lambda$ / nm  & $\lambda_{Theo}$  & Transition\\
      \midrule
      1$\gamma$ & Peak 1  & 384.24 & 780.21  & 780.24 &
      $5S_{1/2} \rightarrow 5P_{3/2} $\\
      \midrule
      \multirow{3}{*}{2$\gamma$} & Peak 1 & 768.45 & 390.12 &  & \\
                                 & Peak 2 & 770.58  & 389.04 & 389.05 (389.09) &
      $5S_{1/2} \rightarrow 5D_{3(5)/2}$\\
                                 & Peak 3 & 771.34 & 388.66 & - & \\
      \bottomrule
  \end{tabular}
 \end{table}
\end{document}
